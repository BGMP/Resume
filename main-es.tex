\documentclass[letterpaper,11pt]{article}

\usepackage{latexsym}
\usepackage[empty]{fullpage}
\usepackage{titlesec}
\usepackage{marvosym}
\usepackage[usenames,dvipsnames]{color}
\usepackage{verbatim}
\usepackage{enumitem}
\usepackage[hidelinks]{hyperref}
\usepackage{fancyhdr}
\usepackage[english]{babel}
\usepackage{tabularx}
\usepackage{graphicx}
\usepackage{fontawesome}
\input{glyphtounicode}

\pagestyle{fancy}
\fancyhf{} % clear all header and footer fields
\fancyfoot{}
\renewcommand{\headrulewidth}{0pt}
\renewcommand{\footrulewidth}{0pt}

% Adjust margins
\addtolength{\oddsidemargin}{-0.5in}
\addtolength{\evensidemargin}{-0.5in}
\addtolength{\textwidth}{1in}
\addtolength{\topmargin}{-.5in}
\addtolength{\textheight}{1.0in}

\urlstyle{same}

\raggedbottom
\raggedright
\setlength{\tabcolsep}{0in}

% Sections formatting
\titleformat{\section}{
  \vspace{-4pt}\scshape\raggedright\large
}{}{0em}{}[\color{black}\titlerule \vspace{-5pt}]

% Ensure that generate pdf is machine readable/ATS parsable
\pdfgentounicode=1

%-------------------------
% Custom commands
\newcommand{\resumeItem}[2]{
  \item\small{
    \textbf{#1}{: #2 \vspace{-2pt}}
  }
}

% Just in case someone needs a heading that does not need to be in a list
\newcommand{\resumeHeading}[4]{
    \begin{tabular*}{0.99\textwidth}[t]{l@{\extracolsep{\fill}}r}
      \textbf{#1} & #2 \\
      \textit{\small#3} & \textit{\small #4} \\
    \end{tabular*}\vspace{-5pt}
}

\newcommand{\resumeSubheading}[4]{
  \vspace{-1pt}\item
    \begin{tabular*}{0.97\textwidth}[t]{l@{\extracolsep{\fill}}r}
      \textbf{#1} & #2 \\
      \textit{\small#3} & \textit{\small #4} \\
    \end{tabular*}\vspace{-5pt}
}

\newcommand{\resumeSubSubheading}[2]{
    \begin{tabular*}{0.97\textwidth}{l@{\extracolsep{\fill}}r}
      \textit{\small#1} & \textit{\small #2} \\
    \end{tabular*}\vspace{-5pt}
}

\newcommand{\resumeSubItem}[2]{\resumeItem{#1}{#2}\vspace{-4pt}}

\renewcommand{\labelitemii}{$\circ$}

\newcommand{\resumeSubHeadingListStart}{\begin{itemize}[leftmargin=*]}
\newcommand{\resumeSubHeadingListEnd}{\end{itemize}}
\newcommand{\resumeItemListStart}{\begin{itemize}}
\newcommand{\resumeItemListEnd}{\end{itemize}\vspace{-5pt}}

%-------------------------------------------
%%%%%%  CV STARTS HERE  %%%%%%%%%%%%%%%%%%%%%%%%%%%%


\begin{document}

%----------HEADING-----------------
\begin{tabular*}{\textwidth}{l@{\extracolsep{\fill}}r}
  \textbf{\href{https://bgm.cl/}{\Large José Benavente}} & Email : \href{mailto:jose@bgm.cl}{jose@bgm.cl}\\
  \href{https://bgm.cl/}{https://bgm.cl} & \includegraphics[width=3mm, height=3mm]{img/github-logo.png} \href{https://github.com/BGMP}{BGMP}\\ & \includegraphics[width=3mm, height=3mm]{img/gitlab-logo.png} \href{https://gitlab.com/BGMP}{BGMP}\\
\end{tabular*}

%-----------EXPERIENCE-----------------
\section{Experiencia Laboral}
\resumeSubHeadingListStart
	\resumeSubheading
		{Universidad del Bío-Bío}{Concepción, Chile}
		{Profesor part-time de Taller de Desarrollo para ingeniería de ejecución informática.}{Abril 2024 --}
	\resumeSubheading
		{Universidad del Bío-Bío}{Concepción, Chile}
		{Profesor ayudante de Taller de Desarrollo para ingeniería de ejecución informática.}{Agosto 2023 -- Enero 2024}
	\resumeSubheading
		{Olimpiada Chilena de Informática (OCI)}{Concepción, Chile}
		{Profesor de programación competitiva en C/C++.}{Abril 2022 -- Junio 2023}
\resumeSubHeadingListEnd
  
 %-----------VOLUNTEERING-----------------
\section{Voluntariado}
  \resumeSubHeadingListStart
      \resumeSubheading
		  {GitHub (\textnormal{\url{https://github.com}})}{California, Estados Unidos}
		  {\href{https://githubcampus.expert/BGMP}{GitHub Campus Expert $\diamond$}}{2023 ---}
		  \resumeItemListStart
			  \item{Fundar y liderar una comunidad de desarrollo de software en mi universidad con el apoyo de GitHub.}
			  \item{Organizar eventos promoviendo el desarrollo colaborativo abiertos a todas las carreras de mi universidad.}
			  \item{Crear espacios inclusivos y seguros para que los estudiantes compartan recursos de programación.}
		  \resumeItemListEnd
      \resumeSubheading
		  {Re-Volt: OpenGL (\textnormal{\url{https://rvgl.org}})}{}
		  {Desarrollador de software, miembro del equipo core de mantenedores}{Enero 2022 ---}
		  \resumeItemListStart
			  \resumeItem{C/C++}{Implementar correcciones de sincronización al aspecto multijugador del juego, así como también resolver errores generales reportados por su comunidad.}
			  \resumeItem{GitHub/GitLab}{Trabajar con un equipo de desarrolladores de todas partes del mundo de manera remota para lanzar actualizaciones del juego, además de colaborar con el versionado de la base de código.}
			  \resumeItem{mdBook}{Escribir documentación para las nuevas versiones del juego.}
			  \item{Interactuar con la comunidad del juego de más de 5000 jugadores para así resolver bugs y agregar nuevas funcionalidades.}
		  \resumeItemListEnd
	 \resumeSubheading
		  {Re-Volt America (\textnormal{\url{https://rva.lat}})}{}
		  {Líder de desarrollo de software}{Mayo 2021 ---}
		  \resumeItemListStart
		  \resumeItem{Ruby on Rails}{Diseñar y desarrollar una aplicación para la comunidad, la cual sirve a miles de usuarios al mes.}
		  \resumeItem{Capistrano}{Implementar flujos de trabajo en Capistrano y GitHub Actions para desplegar a producción}
		  \resumeItem{MongoDB}{Diseñé, implementé y desplegué modelos y relaciones de la base de datos.}
		  \resumeItem{Redis}{Reduje los tiempos de carga de servicios y vistas pesadas en un 300\% utilizando caché con Redis.}
		  \resumeItem{Sentry}{Monté Sentry para el manejo de errores lanzados por la aplicación en producción.}
		  \resumeItem{Rake}{Diseñé y, en el presente, mantengo varios flujos de trabajo de despliegue utilizando Ruby Make (Rake).}
		  \resumeItemListEnd
		\resumeSubheading
		  {Mojang Studios (\textnormal{\url{https://mojang.com}})}{Estocolmo, Suecia}
		  {\href{https://crowdin.com/profile/bgm}{Traductor y revisor de traducciones oficial para el videojuego Minecraft $\diamond$}}{2017 ---}
		  \resumeItemListStart
			  \item{Traducir el videojuego Minecraft del inglés de Estados Unidos al Español de Chile y a otras variantes del inglés, como el inglés del Reino Unido y el inglés de Australia.}
			  \item{Verificar traducciones oficiales para Minecraft y su Launcher, las cuales llegan a miles de millones de usuarios.}
		  \resumeItemListEnd
   \resumeSubHeadingListEnd
   
%-----------EDUCATION-----------------
\section{Educación}
\resumeSubHeadingListStart
	\resumeSubheading
	{Universidad del Bío-Bío}{Concepción, Chile}
	{Ingeniería de Ejecución en Computación e Informática}{Marzo 2020 -- Enero 2024}
\resumeSubHeadingListEnd

%-----------SPEAKING-----------------
\section{Charlas y Talleres}
\resumeSubHeadingListStart
  \resumeSubheading
	  {\href{https://youtu.be/S_qkjY7k_L8?t=2698}{Contribuyendo a Open Source, una guía práctica $\diamond$}}{Santiago, Chile}
		{Fui invitado por GitHub a dar una charla de cómo contribuir a proyectos open-source en DUOC UC.}{04/2024}
  \resumeSubheading
	  {\href{https://bgm.cl/about}{Git/GitHub Workshop $\diamond$}}{Concepción, Chile}
	  {Presenté los fundamentos de Git y GitHub, junto con ejercicios prácticos para todos los asistentes.}{08/2023}
  \resumeSubheading
	  {\href{https://bgm.cl/about}{Ruby on Rails Workshop $\diamond$}}{Concepción, Chile}
	  {Realicé una introducción a Ruby on Rails para el desarrollo de aplicaciones web full-stack.}{09/2023}
\resumeSubHeadingListEnd

%--------CERTIFICATIONS------------
\section{Certificaciones}
\resumeSubHeadingListStart
\item{
	\textbf{GitHub Foundations}{: \url{https://www.credly.com/badges/7cc1c714-3511-4d48-8281-8e23089fad25/public_url}}
}
\resumeSubHeadingListEnd

%--------PROGRAMMING SKILLS------------
\section{Habilidades de Programación}
  \resumeSubHeadingListStart
    \item{
      \textbf{Lenguajes}{: C/C++, Java, Ruby, Python, PHP, Haml, Sass, TeX}
    }
    \item{
      \textbf{Frameworks}{: Ruby on Rails, Liquid, wxPython, Yii}
    }
     \item{
    	\textbf{Cloud}{: DigitalOcean, Azure}
    }
    \item{
      \textbf{Tecnologías y Herramientas}{: Capistrano, Rake, MongoDB, Redis, Maven, Gradle, Guice, PyInstaller, Docker, Git, GitHub Actions, vsftpd}
    }
  \resumeSubHeadingListEnd

%--------LANGUAGES------------
\section{Idiomas}
\resumeSubHeadingListStart
\item{
  \textbf{Español}{ (Nativo)}
}
\item{
  \textbf{Inglés}{ (C2)}
}
\item{
  \textbf{Italiano}{ (B2)}
}
\resumeSubHeadingListEnd

%-------------------------------------------
\end{document}
